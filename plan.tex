\section{Plan}
  \subsection{Possible Privacy-Preserving Laws/Technologies}
    \par User privacy is a fundamental human right. Coercing people into turning the GPS on and granting their Location access to the third-party application all the time is not the right thing to do. One of the proposed solutions is convincing users that all collected data will be encrypted and send to the Government only, anyone else, even though developers cannot know exactly where they are and where they were.
  \subsection{Accuracy of Technologies}
    \par Since the behavior of the virus varies depending on different conditions (e.g. type of surface, temperature or humidity of the environment) \parencite{Plan3}, it is an arduous task to know exactly how long the virus lasts outside the host. A solution for this is when knowing that an infected person went to a specific place in the past, the mobile application would push notification to all the people within 5 kilometers radius with real-time tracking since the day that infected person came to until 72 hours later \parencite{Plan4}. Hence, all notified people would raise the awareness of limiting on surface touching and regularly washing their hands with an alcohol-based hand rub or wash them with soap and water.
    \par According to a study published last year \parencite{Plan1}, a phone is typically able to determine its position with an accuracy between 7 and 13 meters in urban areas. However, it is recommended that people should keep 1.5 meters away from others as much as possible is one of the ways proposed by Social distancing (also called physical distancing) to help slow the spread of viruses \parencite{Plan2}. Therefore, the accuracy of the GPS of a mobile phone radically affects the results of distancing among people even though they are not in that close contact with each other (within 1.5 meters). In order to tackle this problem, what our application could do is sending the data to the Government only then they would decide how these data might be handled: Maybe they would do nothing, or give warning to the people who were in close contact with the infected person about quarantining at home in 14 days. Furthermore, cooperating with third-party services such as Translink would make a great contribution to the tracking system of our project. For instance, if there was a person who was infected with COVID-19 came to a bus, Translink could get access to his/her Go card in order to track which bus they took in recent days then the company would sterilize that specific bus and send back information to us to notify all associated people who took the same bus with the COVID-19 patient. Another feasible solution to tackle the problem of accuracy is by using short-range Bluetooth signals from other’s smartphones. The biggest advantage of this application compared to other existing ones is the combination of location services and Bluetooth signals. With GPS, the system can track the areas where the person who tested positive for the virus came with the purpose of minimizing the spread of the virus outside the host. Moreover, thanks to Bluetooth, the system could identify people who were in close contact with that patient and then push a notification to them about what to do next. Supposing when a person is required to self-quarantine at home, the application would frequently interact with that person via Q\&A sections and provide emergency contact to medical services in case of unwell health conditions. In addition, the GPS, as well as Bluetooth signal, are needed to be accessed all the time, included in and after lockdown to reduce the pandemic outbreak.
  \subsection{Conclusion}
    \par The purpose of this application is health communication. However, location accessibility is the key to the success of this application. Our application could be gotten the most to prevent the pandemic if there were cooperation among inhabitants to get access to their location for real-time tracking. The situation was put on high alert if there was a person who infected with COVID-19 did not turn GPS and Location Services on their mobile phone and then entered public places (e.g. Supermarket, City center). In consequence, whoever entered those places within 3 days might acquire the virus after touching contaminated objects or surfaces.

