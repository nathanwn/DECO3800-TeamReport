\section{Stakeholders}
  \subsection{Users}
    \par It is obvious that this app is designed for everyone around the world. All you need is a smartphone or a wearable device to install the app. At the time when the pandemic seems to be out of control, this app is a considerable solution that helps users gain more understanding of their health conditions. It cannot be denied that when the scene is extremely dangerous, social distancing is one of the best solutions to prevent the virus from spreading. However, there are some situations that people must go outside such as grocery shopping or medical appointments \parencite{Stake1}. It is impossible to lock yourself inside your house 24/7. This app will help to track all other people that the user has been in contact with when they were outside. Although the number of people might be modest thanks to social distancing, prevention is still compulsory. Moreover, there are specific jobs that people cannot work from home. People may be working in public services, in retail, in vital parts of the manufacturing sector or for the governments \parencite{Stake2}. These paramount jobs cannot be converted instantly into homeworking. Therefore, there will still be several people heading out for works facing a high risk of infection. This app will support those people by storing the data of everyone they have met at the workplace or on the way to work.
  \subsection{Doctors/Nurses}
    \par Doctors and nurses probably benefit a lot from this app. The more we prevent the virus from spreading, the less the pressure on them. In countries which are facing the pandemic like Italy, doctors have to make heartbreaking decisions about who receives the treatment and who is abandoned \parencite{Stake3}. Furthermore, because of the sudden rise in the number of infected cases, there is not enough equipment for health workers \parencite{Stake4}. Consequently, many of them are risking their lives curing patients. Healthcare workers are influential in the war with pandemics and protecting them should be the number one priority. If they are infected, they must be isolated for at least 14 days, which will deplete the already exhausted workforce \parencite{Stake5}. By reducing the number of cases, doctors will be in contact with fewer infected patients. Finally, the situation of the pandemic is serious does not mean that we neglect patients with other sicknesses.
  \subsection{Governments/Health Departments/Health Organizations}
    \par It cannot be denied that governments play an important role in stopping the spread of pandemics. They are the frontline and their decisions highly influencing the fight against the epidemics. The governments have adopted some regulations like social distancing to protect and support their people. It is the governments’ responsibility to track the virus and put an end to it. With this app, they can easily find out who the infected patient has been in contact with instead of inquiring him/her. The results came out from investigating infected patients might be inaccurate due to the unstable health condition of them. Another problem that the governments and health departments are facing during the pandemic outbreak is the shortage of hospital beds. For instance, in the US, many hospitals have become overloaded \parencite{Stake6}. Based on overseas experience, about one-fifth of people infected with COVID-19 need hospital care and 5\% will need intensive care because of their critical health status \parencite{Stake3}. Therefore, if the coronavirus keeps spreading with the ongoing rate, the number of infected patients will exceed the number of available beds soon. In response, the governments have built more field hospitals which only cure COVID-19 patients. As proof, a dedicated coronavirus field hospital is being built in Canberra which could cost more than \$23 million \parencite{Stake7}. This application helps to reduce the number of infected cases also means that governments can save a lot of money.
    \par Our app is designed to track and store all the data of people whom users have been in contact during the plague. In other words, our application is mediated between governments and users. Governments and Health Department have access to the database so that they can select which users are most likely to be infected. After that, our application will be the means of communication for the government to connect with those users.
    \par This application also needs the cooperation of the Health Organization or Health Department to obtain all the information about the current pandemic (namely symptoms, how it spreads, how to protect yourself, etc.). With these data, the application can adapt and use the most effective tracking method.
  \subsection{Businesses}
    \par Not only a health concern, COVID-19 is seriously impacting business and the economy. It cannot be denied that this pandemic is affecting many aspects of the economy. According to data from the Australian Bureau of Statistics (ABS), approximately 90\% of businesses in Australia were expected to be impacted in the near future if the current situation continues \parencite{Stake2-1}. It can be said that the hospitality industry is the hardest hit. The severe crises that the hospitality industry is suffering from are worse than those of 9/11, SARS, and the financial crisis in 2008 \parencite{Stake2-2}. Many flights have been canceled, leading to a loss of aviation in every country. Restaurants and hotels are closed due to social distancing rules and the fact that no one dares to go out during the pandemic outbreak. There are some restaurants that are still open, but the government only allows them to serve takeaway food. This industry contains lots of family businesses like eateries, coffee shops, pubs, bistros, etc. Being a family business means that things will get even tougher as the demand falls and change in the supply chain \parencite{Stake2-2}. Some of them might totally shut down and wait for this virus to be contained so that they can continue their business.
    \par Apart from the hospitality industry, there are many other forms of the economy that are suffering no less. The number of unemployed people in the US have increased rapidly during the outbreak. In the first week of April, weekly total of new unemployment claims in this country reached nearly 7 million \parencite{Stake2-3}. Economists also warned that since the Great Depression in the 1930s, the world economy has never stagnated like it is now \parencite{Stake2-4}. Due to social distancing, people are asked to stay inside and only go out for necessary purposes. Consequently, the little demand for oil leads to a sharp drop in oil prices. For instance, oil prices in the US are negative for the first time the history \parencite{Stake2-3}. Coronavirus also changes how traditional commerce works. Letting people make transactions at banks is almost possible because according to social distancing restrictions, people must stay 1.5 meters away from others. Compliance with these rules while in the bank is impractical since banks are always crowded with people. Moreover, this virus can last on the surface of objects like bills, banknotes, or paper money. To tackle this issue, many banks have encouraged their customers to use digital banking instead of in-person banking or physical exchanges \parencite{Stake2-5}. Digital banking is feasible, but initially, it will face many difficulties such as training their staff to use digital tools. Furthermore, this virus is also the reason why many new companies afraid to enter the market.
    \par It is certainly true that when the disease outbreaks are severe, social distancing is the best method that we had to limit the interaction between people. But when the situation gets better, restrictions must be lifted so that businesses can open again. Although it is possible that the spread of this virus is under control, the government should not be too neglected because the possibility of reinfection of COVID-19 is still unknown. Therefore, our model is a reliable method to supervise the spread of this virus. Businesses can operate normally again if they agree to utilize our model and commit that they only serve customers who have the application installed on their digital device. By this approach, people will be more secure when they travel to places like restaurants, supermarkets, shopping malls, etc. and if the disease happens again, it will be easier for the government to find its origins.
    \subsection{Organizations}
    \par To achieve this model, there must be a way to encourage people to install it. If there is no way to attract people, it will be challenging to convince them to download the application. One is example is that the application named COVIDSafe, which is launched by the Australian government, only reached roughly 30\% of smartphone users age 14 and over after 18 days \parencite{Stake2-6}. It seems that one of the best solutions is to cooperate with organizations that are offering supports during the lockdown. There are lots of support packages out there for people during the pandemic outbreak:
      \begin{itemize}
        \item University of Technology Sydney offers support funds up to 15 million Australian Dollars to help students who are facing financial difficulties \parencite{Stake2-7}.
        \item Queensland government also has support packages for workers who have lost their job because of COVID-19 \parencite{Stake2-8}.
        \item Banks also make a move to help customers and businesses. To illustrate, BankSA has support for credit card and personal loan customers \parencite{Stake2-9}.
        \item One of the major telecom companies in Australia is Telstra offers supplementary data for users (register for 25GB of bonus data, unlimited data at home, unlimited call, etc.) \parencite{Stake2-10}.
      \end{itemize}
    \par By working with these organizations that have been mentioned above and more, it is more convincing for people to download the application. With this application installed on their digital devices, users will be given priority when registering for these support packages. Moreover, users can easily seek help via the application. There will be hotlines to health support departments, hotlines for people with disability who needs helps during pandemic outbreak and hotlines for those who are facing domestic violence and similar anxieties.



