\section{Team Reflection}
  \par There is no doubt that COVID-19 has made a lot of impact to our team's efficiency. It can be said that in order for a team to perform at its best, there is a need for direct interaction between team members. Fortunately, five out of six members in our team live in the same student accommodation which is our advantage over other teams. We could easily meet each other and discuss when there was a need to revise the project. But it was not for that reason that we did not have online meetings every week to update the status of the assignment. We had a weekly meeting on Zoom after every studio class to further discuss our ideas. Even in weeks without studio classes, we still tried to hold a meeting by ourselves so that we could accelerate the progress of our project. Our team also created a shared folder on Google Drive and a private channel on Slack. The folder was where we uploaded all the important files namely reports, clips of the presentation and notes. The Slack channel was our main means of communication. All the crucial announcements would be posted on this channel so that the team members could keep up with the flow of the project. Besides, we used the channel to share recent articles, reports or any literature that were relevant to our problem space and solution. 
  \par At the beginning of each meeting, we would make time for each person to present shortly what they had researched in that week. We utilized the whiteboard and screen sharing on Zoom to take notes during the meetings. Once everyone in the team was done presenting, we moved on to solving problems still present in our model. At the same time, we also revised the comments from our tutors and classmates on the project. These comments were very valuable as they provided us with a more objective perspective on the issues. Moreover, they helped us better understand the users’ needs and expectations. Before any meeting ending, we divided the work based on the requirements of the next assessment item. A team-only deadline was always set to guarantee that our work would not be delayed. In the end, all the notes that had been taken during the meetings were posted to our Slack channel for later reviewing. Although our project is far from being perfect and there are still many issues that we have to tackle, everyone in the team has put in their best effort to develop the project up to the point where it is right now. Healthcare, which is the primary topic that our application concerns, is not our major yet we still did try to gain as much knowledge as possible in this field and of course in the software technical field as well. The evidence is that each member was fully responsible for the part they were assigned. We also adhered to the rules set out to make sure the group works the most effectively.
  \par Nevetheless, in spite of our appropriate and logical team work etiquette as well as everyone's devotion to the success of the project, there were still a few difficulties that went against our will:
    \begin{itemize}
      \item The fact that English is everyone's second language occasionally made it fairly difficult to express and undestand each other's ideas due to the difference in our speaking accents and ways of phrasing. To overcome this problem, all of us had to pay close attention and be patient whenever a member was speaking up his ideas or opinions. 
      \item As the semester had moved fully online, networking issues were undoubtedly another difficulty we had to face. Sometimes someone in the team could not catch up with the online meeting discussion because of weak Internet stability. To solve this, we either needed to repeat things multiple times or recorded the entire meeting for later review. In the worst case which is not having an Internet connection (which happened to the five of us who live in the same student accommodation as the Internet provider was going through some maintenace at the time of studio meeting), we had to use our 4G mobile data as an alternative.
      \item Although having five out of six members living under the same accommodation has already been a great advantage for our team, communications via online media with the one other member were sometimes challenging as online messages obviously do not get acknowledged immediately similar to the way that face-to-face conversations do. Therefore, we had to exchange mobile phone numbers in case the other member needed to be notified of any urgent information.
    \end{itemize}
  \par All in all, despite all of the difficulties and challenges our team had to face and find a way to overcome throughout the development of the project, we all feel satisfied with how we have functioned as a team as well as the final solution that we have managed to deliver. We are all confident to admit that our operation was acceptably seamless, which lives up to the name of our team - United.   
